\documentclass[11pt, a4paper]{article}
%\usepackage{geometry}
\usepackage[inner=1.5cm,outer=1.5cm,top=2.5cm,bottom=2.5cm]{geometry}
\pagestyle{empty}
\usepackage{graphicx}
\usepackage{fancyhdr, lastpage, bbding, pmboxdraw}
\usepackage[usenames,dvipsnames]{color}
\definecolor{darkblue}{rgb}{0,0,.6}
\definecolor{darkred}{rgb}{.7,0,0}
\definecolor{darkgreen}{rgb}{0,.6,0}
\definecolor{red}{rgb}{.98,0,0}
\usepackage[colorlinks,pagebackref,pdfusetitle,urlcolor=darkblue,citecolor=darkblue,linkcolor=darkred,bookmarksnumbered,plainpages=false]{hyperref}
\renewcommand{\thefootnote}{\fnsymbol{footnote}}

\pagestyle{fancyplain}
\fancyhf{}
\lhead{ \fancyplain{}{Machine Learning} }
%\chead{ \fancyplain{}{} }
\rhead{ \fancyplain{}{Spring, 2017} }
%\rfoot{\fancyplain{}{page \thepage\ of \pagafaeref{LastPage}}}
\fancyfoot[RO, LE] {page \thepage\ of \pageref{LastPage} }
\thispagestyle{plain}

%%%%%%%%%%%% LISTING %%%
\usepackage{listings}
\usepackage{caption}
\DeclareCaptionFont{white}{\color{white}}
\DeclareCaptionFormat{listing}{\colorbox{gray}{\parbox{\textwidth}{#1#2#3}}}
\captionsetup[lstlisting]{format=listing,labelfont=white,textfont=white}
\usepackage{verbatim} % used to display code
\usepackage{fancyvrb}
\usepackage{acronym}
\usepackage{amsthm}
\VerbatimFootnotes % Required, otherwise verbatim does not work in footnotes!



\definecolor{OliveGreen}{cmyk}{0.64,0,0.95,0.40}
\definecolor{CadetBlue}{cmyk}{0.62,0.57,0.23,0}
\definecolor{lightlightgray}{gray}{0.93}



\lstset{
%language=bash,                          % Code langugage
basicstyle=\ttfamily,                   % Code font, Examples: \footnotesize, \ttfamily
keywordstyle=\color{OliveGreen},        % Keywords font ('*' = uppercase)
commentstyle=\color{gray},              % Comments font
numbers=left,                           % Line nums position
numberstyle=\tiny,                      % Line-numbers fonts
stepnumber=1,                           % Step between two line-numbers
numbersep=5pt,                          % How far are line-numbers from code
backgroundcolor=\color{lightlightgray}, % Choose background color
frame=none,                             % A frame around the code
tabsize=2,                              % Default tab size
captionpos=t,                           % Caption-position = bottom
breaklines=true,                        % Automatic line breaking?
breakatwhitespace=false,                % Automatic breaks only at whitespace?
showspaces=false,                       % Dont make spaces visible
showtabs=false,                         % Dont make tabls visible
columns=flexible,                       % Column format
morekeywords={__global__, __device__},  % CUDA specific keywords
}

%%%%%%%%%%%%%%%%%%%%%%%%%%%%%%%%%%%%
\begin{document}
\begin{center}
{\Large \textsc{Machine Learning}}
\end{center}
\begin{center}
Spring 2017
\end{center}
%\date{September 26, 2014}

\begin{center}
\rule{6in}{0.4pt}
\begin{minipage}[t]{.75\textwidth}
\begin{tabular}{llcccll}
\textbf{Contact:} & Patrick Beukema & & &  & \textbf{Time:} & W 12:00 -- 1:00 \\
\textbf{Email:} &  \href{plb23@pitt.edu}{plb23@pitt.edu} & & & & \textbf{Place:} & Mellon Institute 115
\end{tabular}
\end{minipage}
\rule{6in}{0.4pt}
\end{center}
\vspace{.5cm}
\setlength{\unitlength}{1in}
\renewcommand{\arraystretch}{2}

\noindent\textbf{Course Pagse:}

\begin{enumerate}

\item Course website: \url{http://github.com/pbeukema/ML_CNBC}

\item All code referenced in text: \url{http://github.com/probml/pmtk1/tree/master/pmtk/bookCode}


\item Slack page for quetions about material/exercises: \url{mlcnbc.slack.com}

\end{enumerate}

\noindent\textbf{Text:} %\footnotemark

Machine Learning: A Probabilistic Perpspective. Each week we will read, do exercises, and discuss one chapter of this text. \href{https://www.cs.ubc.ca/~murphyk/MLbook/}{https://www.cs.ubc.ca/~murphyk/MLbook/}

\vskip.15in
\noindent\textbf{Schedule \& Exercises:}
\begin{enumerate}
\item Introduction 1/11
\begin{itemize}
\item Write implementation of KNN, exercises: 1.1-1.3

\end{itemize}
\item Probability 1/18
\begin{itemize}
\item Exercises: 2.1-2.5, 2.12, 2.17 (Prove, and also show with simulation)

\end{itemize}
\item Generative models for discrete data 1/25
\begin{itemize}
\item Write implementation of Naive Bayes, exercises 3.6, 3.10, 3.18.
\end{itemize}
\item Gaussian models 2/1
\begin{itemize}
\item Write implementation of LDA/QDA, exercises 4.17, 4.12. 
\end{itemize}
\item Bayesian statistics 2/8 
\begin{itemize}
\item Exercises
\end{itemize}
\item Frequestist statistics 2/15
\begin{itemize}
\item Exercises
\end{itemize}
\item Linear Regression 2/22
\begin{itemize}
\item Exercises
\end{itemize}
\item Logistic Regression 3/1
\begin{itemize}
\item Exercises
\end{itemize}
\item Generalized linaer models and the exponential family 3/8
\begin{itemize}
\item Exercises
\end{itemize}
\item Directed graphical models 3/15
\begin{itemize}
\item Exercises
\end{itemize}
\item Mixture models and the EM algorithm 3/22
\begin{itemize}
\item Exercises
\end{itemize}
\item Latent linear models 3/29
\begin{itemize}
\item Exercises
\end{itemize}
\item Sparse linear models 4/5
\begin{itemize}
\item Exercises
\end{itemize}

\item Kernels 4/12
\begin{itemize}
\item Exercises
\end{itemize}
\item Gaussian processes 4/19
\begin{itemize}
\item Exercises
\end{itemize}
\end{enumerate} 



\vskip.15in
\noindent\textbf{Course objectives:}  The idea behind this course is to gain exposure and learn the fundamentals of machine learning through weekly discussions and exercises. Secondary goal is to republish the code that is used throughout the text and excersises in the form of jupyter notebooks in python. The motivation for that is (1) Current implemntations are in Matlab, which some people lack access to and (2) more learning will happen doing things from scratch. 



\vskip.15in
\noindent\textbf{Prerequisites:} Calculus, linear algebra, probability. 


\vskip.15in


\vskip.15in
\noindent\textbf{Resources \& Recommendations:}  
\begin{itemize}
\item Learning python: \href{https://learnpythonthehardway.org}{learnpythonthehardway.org}
\item Programming:  \href{https://jupyter.org/}{jupyter.org} notebooks\footnote{Kyle Dunovan's \href{https://github.com/dunovank/jupyter-themes}{themes} make your notebooks pretty }
\item  Editor: \href{https://atom.io/}{atom.io} 

\end{itemize}

\vskip.15in



%%%%%% END 
\end{document} 